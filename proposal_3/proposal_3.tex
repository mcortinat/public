% ===== Document class & geometry =====
\documentclass{article}
\usepackage[top=1in, bottom=1in, left=1in, right=1in]{geometry}

% ===== Encoding & language =====
\usepackage[utf8]{inputenc} % <-- swap utf8x -> utf8
\usepackage[english]{babel}

% ===== Math =====
\usepackage{amsmath, amsfonts, amssymb, bm}
\usepackage{amsthm}

% ===== Tables & floats =====
\usepackage[table]{xcolor}
\usepackage{array, tabularx, booktabs, longtable, makecell, multirow}
% \usepackage{tabu} % <-- REMOVE: abandoned & buggy
\usepackage{caption}
\usepackage{subcaption} % keep this; DO NOT also load floatrow
% \usepackage[capposition=top]{floatrow} % <-- REMOVE: conflicts with subcaption

% ===== Graphics =====
\usepackage{graphicx}
\usepackage{pdflscape}
\usepackage{pdfpages}
\usepackage{rotating}
\usepackage{epstopdf}

% ===== Text, layout, headers/footers =====
\usepackage{setspace}
\usepackage{ragged2e}
\usepackage{fancyhdr}
\usepackage[normalem]{ulem}
\usepackage{enumerate}
\usepackage{framed}
\usepackage{comment}
\usepackage{titlesec}
\usepackage{titletoc} % load once
\usepackage{accents}
\usepackage{stackengine}
\usepackage{footmisc} % choose ONE footnote package
% \usepackage{footnote} % <-- REMOVE to avoid conflicts

% ===== Links (load near the end) =====
\usepackage[colorlinks=true,linkcolor=darkgray,citecolor=darkgray,urlcolor=darkgray,anchorcolor=darkgray]{hyperref}

% ===== Bibliography (natbib + BibTeX) =====
\usepackage[sort&compress]{natbib}
% \usepackage{bibunits} % <-- Usually unnecessary; remove unless you truly need per-section bibs

% ===== Colors & watermark (optional) =====
\definecolor{lightgray}{RGB}{220,220,220}
\definecolor{dimgray}{RGB}{105,105,105}
% \usepackage[printwatermark]{xwatermark} % optional; safe to keep disabled

% ===== Section formats =====
\titlespacing{\section}{.2pt}{1ex}{1ex}
\titleformat{\section}{\centering \normalsize \bfseries}{\thesection.}{0em}{}
\renewcommand{\thesection}{\arabic{section}}
\titleformat{\subsection}{\flushleft \normalsize \bfseries}{\thesubsection}{0em}{}
\renewcommand{\thesubsection}{\arabic{section}.\arabic{subsection}}

% ===== Theorems =====
\newtheoremstyle{mytheoremstyle}
{\topsep}{\topsep}{\color{black}}{}{\itshape\color{dimgray}}{.}{.5em}{}
\theoremstyle{mytheoremstyle}
\newtheorem{assumption}{Assumption}
\renewcommand\theassumption{\arabic{assumption}}
\newtheorem{assumptiona}{Assumption}
\renewcommand\theassumptiona{\arabic{assumptiona}a}
\newtheorem{assumptionb}{Assumption}
\renewcommand\theassumptionb{\arabic{assumptionb}b}
\newtheorem{assumptionc}{Assumption}
\renewcommand\theassumptionc{\arabic{assumptionc}c}
\newtheorem{lemma}{Lemma}
\newtheorem{proposition}{Proposition}
\newtheorem{corollary}{Corollary}

% ===== Math operators & handy commands =====
\DeclareMathOperator{\cov}{Cov}
\DeclareMathOperator{\sign}{sgn}
\DeclareMathOperator{\var}{Var}
\DeclareMathOperator{\plim}{plim}
\DeclareMathOperator*{\argmin}{arg\,min}
\DeclareMathOperator*{\argmax}{arg\,max}
\newcommand\independent{\protect\mathpalette{\protect\independenT}{\perp}}
\def\independenT#1#2{\mathrel{\rlap{$#1#2$}\mkern2mu{#1#2}}}
\newcommand{\overbar}[1]{\mkern 1.5mu\overline{\mkern-1.5mu#1\mkern-1.5mu}\mkern 1.5mu}
\newcommand{\equald}{\ensuremath{\overset{d}{=}}}
\newcommand\barbelow[1]{\stackunder[1.2pt]{$#1$}{\rule{1ex}{.085ex}}}

% ===== Captions =====
\captionsetup[table]{skip=10pt}
\captionsetup[figure]{labelfont={bf},name={Figure},labelsep=period}
\renewcommand{\thefigure}{\arabic{figure}}
\captionsetup[table]{labelfont={bf},name={Table},labelsep=period}
\renewcommand{\thetable}{\arabic{table}}

% ===== Paragraphing, spacing, column types =====
\setlength{\parindent}{24pt} % (you had both 22pt and 24pt—choose one)
\setlength{\parskip}{5pt}
\newcolumntype{L}[1]{>{\raggedright\let\newline\\\arraybackslash\hspace{0pt}}m{#1}}
\newcolumntype{C}[1]{>{\centering\let\newline\\\arraybackslash\hspace{0pt}}m{#1}}
\newcolumntype{R}[1]{>{\raggedleft\let\newline\\\arraybackslash\hspace{0pt}}m{#1}}

% ===== REMOVE the risky patch (likely cause of \endgroup) =====
% \makeatletter
% \pretocmd\start@align{ \let\everycr\CT@everycr \CT@start }{}{}
% \apptocmd{\endalign}{\CT@end}{}{}
% \makeatother


\begin{document}


\title{\Large \textbf{ECON 603 - Research Proposal III}}

\author{Magdalena Cortina\thanks{Email: mcortinat@tamu.edu.}} 
\date{\today}

\maketitle
\thispagestyle{empty} 
\doublespacing
\thispagestyle{empty} 

\vspace{-10mm}
\pagenumbering{arabic}

\doublespacing

\section{ Research question}

In the past two decades, Chile has experienced a sharp increase in immigration, primarily from Venezuela, Haiti, Peru, and Colombia. During the same period, crime rates have also risen, and the nature of crime has changed, becoming more violent and including new forms of organized and drug-related offenses. The media has often portrayed immigrants as responsible for this shift, shaping public perception and potentially influencing policy attitudes \citep{valenzuela2019media}. Does immigration increase or change the crime patterns in Chile? Are municipalities with higher immigration inflows experiencing greater increases in crime rates? Are immigrants overrepresented among perpetrators of violent or “new” crimes? How does media coverage influence public perception of immigrants and their association with crime?

\section{ Economic Framework and Empirical Design}

Individuals (immigrants and natives) make location and employment decisions under economic constraints (e.g., wages, legal status, housing access). Policymakers and the public react to perceived changes in local crime, often informed by media coverage. 
Who makes the choices? Immigrants decide where to settle; natives and firms adjust behavior based on perceived safety or labor competition.
What are the constraints? Labor market opportunities, local security, and social acceptance.
How do they interact? Through neighborhood-level dynamics, competition, discrimination, and exposure to media narratives.

To study the relationship between immigration and crime, I propose an empirical strategy using a municipality-by-year panel dataset for Chile over the period 2005--2023. The main estimating equation is:

\begin{equation}
	Crime_{mt} = \alpha + \beta \, Immigration_{mt} + \mu_m + \tau_t + X_{mt}'\gamma + \varepsilon_{mt},
\end{equation}

\noindent where $Crime_{mt}$ denotes the crime rate (per 100,000 inhabitants) in municipality $m$ at time $t$, and $Immigration_{mt}$ represents the share of immigrants in the local population. Municipality fixed effects $(\mu_m)$ absorb time-invariant characteristics such as geography and historical violence, while year fixed effects $(\tau_t)$ control for national trends in crime and policy. The vector $X_{mt}$ includes controls such as unemployment, poverty, education, and police presence.

The coefficient $\beta$ captures whether increases in immigration are associated with changes in local crime. However, since immigrants may choose to move to safer or more economically dynamic areas, OLS estimates may be biased. To address this, I propose an instrumental variable (IV) strategy exploiting \emph{pre-existing settlement patterns} by nationality: a shift-share instrument constructed as the interaction between historical immigrant shares (from the 2002 Census) and national inflows by origin in later years.

A second part of the analysis will explore the role of media influence. Using textual analysis of newspaper and television reports, I will construct a measure of anti-immigrant sentiment and examine how media exposure affects public perceptions of crime. Combining this with survey data (e.g., from CEP or UDP), I can test whether perceived crime diverges from actual reported crime, and whether this divergence correlates with local media coverage intensity.


\section{ Data}

Data on crime for Chile is available publicly and it includes the type of crime, age, gender and nationality of the victim and criminal, for every month and location (municipality level) starting in 2000. Immigration data is also public and available 
on the Census and in immigration registries.

\bigskip

\noindent \textbf{[JLG: It would be good to embed this in an economic framework. In addition, it would be good to relate it to a large literature examining the impact of migration within South America. But this could definitely be a fruitful project if the appropriate dataset is available.]}

\vspace{3em}

\noindent \textbf{[AZ: I agree with Jorge. I am not sure if recent work in Columbia studies the effect on crime specifically. Are there other policies in Chile that could provide other ideas for empirical design? For example, quotas or permits?]}

\bibliographystyle{apalike}
\bibliography{literature}

\end{document}
