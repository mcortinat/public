% ===== Document class & geometry =====
\documentclass{article}
\usepackage[top=1in, bottom=1in, left=1in, right=1in]{geometry}

% ===== Encoding & language =====
\usepackage[utf8]{inputenc} % <-- swap utf8x -> utf8
\usepackage[english]{babel}

% ===== Math =====
\usepackage{amsmath, amsfonts, amssymb, bm}
\usepackage{amsthm}

% ===== Tables & floats =====
\usepackage[table]{xcolor}
\usepackage{array, tabularx, booktabs, longtable, makecell, multirow}
% \usepackage{tabu} % <-- REMOVE: abandoned & buggy
\usepackage{caption}
\usepackage{subcaption} % keep this; DO NOT also load floatrow
% \usepackage[capposition=top]{floatrow} % <-- REMOVE: conflicts with subcaption

% ===== Graphics =====
\usepackage{graphicx}
\usepackage{pdflscape}
\usepackage{pdfpages}
\usepackage{rotating}
\usepackage{epstopdf}

% ===== Text, layout, headers/footers =====
\usepackage{setspace}
\usepackage{ragged2e}
\usepackage{fancyhdr}
\usepackage[normalem]{ulem}
\usepackage{enumerate}
\usepackage{framed}
\usepackage{comment}
\usepackage{titlesec}
\usepackage{titletoc} % load once
\usepackage{accents}
\usepackage{stackengine}
\usepackage{footmisc} % choose ONE footnote package
% \usepackage{footnote} % <-- REMOVE to avoid conflicts

% ===== Links (load near the end) =====
\usepackage[colorlinks=true,linkcolor=darkgray,citecolor=darkgray,urlcolor=darkgray,anchorcolor=darkgray]{hyperref}

% ===== Bibliography (natbib + BibTeX) =====
\usepackage[sort&compress]{natbib}
% \usepackage{bibunits} % <-- Usually unnecessary; remove unless you truly need per-section bibs

% ===== Colors & watermark (optional) =====
\definecolor{lightgray}{RGB}{220,220,220}
\definecolor{dimgray}{RGB}{105,105,105}
% \usepackage[printwatermark]{xwatermark} % optional; safe to keep disabled

% ===== Section formats =====
\titlespacing{\section}{.2pt}{1ex}{1ex}
\titleformat{\section}{\centering \normalsize \bfseries}{\thesection.}{0em}{}
\renewcommand{\thesection}{\arabic{section}}
\titleformat{\subsection}{\flushleft \normalsize \bfseries}{\thesubsection}{0em}{}
\renewcommand{\thesubsection}{\arabic{section}.\arabic{subsection}}

% ===== Theorems =====
\newtheoremstyle{mytheoremstyle}
{\topsep}{\topsep}{\color{black}}{}{\itshape\color{dimgray}}{.}{.5em}{}
\theoremstyle{mytheoremstyle}
\newtheorem{assumption}{Assumption}
\renewcommand\theassumption{\arabic{assumption}}
\newtheorem{assumptiona}{Assumption}
\renewcommand\theassumptiona{\arabic{assumptiona}a}
\newtheorem{assumptionb}{Assumption}
\renewcommand\theassumptionb{\arabic{assumptionb}b}
\newtheorem{assumptionc}{Assumption}
\renewcommand\theassumptionc{\arabic{assumptionc}c}
\newtheorem{lemma}{Lemma}
\newtheorem{proposition}{Proposition}
\newtheorem{corollary}{Corollary}

% ===== Math operators & handy commands =====
\DeclareMathOperator{\cov}{Cov}
\DeclareMathOperator{\sign}{sgn}
\DeclareMathOperator{\var}{Var}
\DeclareMathOperator{\plim}{plim}
\DeclareMathOperator*{\argmin}{arg\,min}
\DeclareMathOperator*{\argmax}{arg\,max}
\newcommand\independent{\protect\mathpalette{\protect\independenT}{\perp}}
\def\independenT#1#2{\mathrel{\rlap{$#1#2$}\mkern2mu{#1#2}}}
\newcommand{\overbar}[1]{\mkern 1.5mu\overline{\mkern-1.5mu#1\mkern-1.5mu}\mkern 1.5mu}
\newcommand{\equald}{\ensuremath{\overset{d}{=}}}
\newcommand\barbelow[1]{\stackunder[1.2pt]{$#1$}{\rule{1ex}{.085ex}}}

% ===== Captions =====
\captionsetup[table]{skip=10pt}
\captionsetup[figure]{labelfont={bf},name={Figure},labelsep=period}
\renewcommand{\thefigure}{\arabic{figure}}
\captionsetup[table]{labelfont={bf},name={Table},labelsep=period}
\renewcommand{\thetable}{\arabic{table}}

% ===== Paragraphing, spacing, column types =====
\setlength{\parindent}{24pt} % (you had both 22pt and 24pt—choose one)
\setlength{\parskip}{5pt}
\newcolumntype{L}[1]{>{\raggedright\let\newline\\\arraybackslash\hspace{0pt}}m{#1}}
\newcolumntype{C}[1]{>{\centering\let\newline\\\arraybackslash\hspace{0pt}}m{#1}}
\newcolumntype{R}[1]{>{\raggedleft\let\newline\\\arraybackslash\hspace{0pt}}m{#1}}

% ===== REMOVE the risky patch (likely cause of \endgroup) =====
% \makeatletter
% \pretocmd\start@align{ \let\everycr\CT@everycr \CT@start }{}{}
% \apptocmd{\endalign}{\CT@end}{}{}
% \makeatother





\begin{document}
	
\title{\Large \textbf{ECON 603 - Research Proposal II}}
\author{Magdalena Cortina\thanks{Email: mcortinat@tamu.edu.}} 
\date{\today}

\maketitle
\thispagestyle{empty} 
\doublespacing
\thispagestyle{empty} 

\vspace{-10mm}
\pagenumbering{arabic}

\doublespacing

\section{ Research question}

\noindent What is the impact of the implementation of the enforcement of payments of child support debt?

This question is important for two main reasons:

\begin{enumerate}
    \item \textbf{Negative effects on children.} There's evidence that increased child support payments can improve the academic achievement of elementary school children even more than income from other sources \citet{knox_effects_1996, nepomnyaschy_child_2012}. 
    
    \item \textbf{Negative effects on mothers.} Most child support creditors are women. Parenting is a shared responsibility, yet when fathers are absent or fail to pay child support, the entire economic burden falls on mothers. This reduces their ability to invest in themselves due to the higher costs of raising children.
\end{enumerate}

In Chile, 84\% of the child support ordered by family courts goes unpaid, and 96\% of debtors are men. In 2022, the national registry of child support debtors was created, and in 2023 the enforcement law was implemented. This law allows Family Courts to investigate and automatically seize funds from bank and investment accounts. By May 2025, family courts had ordered the payment of over USD 2.6 billion, benefiting approximately 282,000 families.

Given this magnitude, it is important to understand how these transfers are affecting both children and mothers. In this proposal, I focus on mothers: specifically, whether enforcement has improved their ability to invest in themselves. Are they now investing more in education, starting businesses, or improving their access to credit? \citet{aizer_impact_2005} finds that stronger enforcement increases average education of mothers, increases maternal prenatal care and investment in children.

\section{ Economic Framework and Empirical Design}
Before enforcement, mothers face a tighter budget constraint because they do not receive the transfers owed to them, having to pay for childcare, food, education, and other child-related expenses on their own. Fathers may strategically default due to weak enforcement. With the reform, constraints change: mothers have greater disposable income, while fathers face higher enforcement costs and reduced opportunities to default. Courts shift bargaining power towards mothers by guaranteeing enforcement. This could alter intra-household bargaining, investment decisions, and labor supply choices.

A possible empirical design for this project is a \textit{difference-in-differences} approach: comparing outcomes of eligible mothers before and after 2023 (the reform), relative to mothers without child support orders. I will also explore heterogeneity by baseline debt size, mother’s education, and income.

\section{ Data}
The \textit{Chilean Household Finance Survey} (Encuesta Financiera de Hogares), conducted every three years, is a promising data source. The latest round (2024) was fielded after the bill’s implementation. Although the data are not yet available, I am currently contacting researchers at the Central Bank of Chile, which administers the survey, to determine whether it captures variables related to child support payments, debt enforcement, and household investments.

\bibliographystyle{apalike}
\bibliography{child_support}

\end{document}
