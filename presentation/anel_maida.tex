\documentclass[aspectratio=169]{beamer}


% ===== THEME =====
\usetheme{Madrid}
\usecolortheme{default}
\setbeamertemplate{navigation symbols}{}
\setbeamertemplate{footline}[frame number]

% ===== FONTS & MATH =====
\usepackage[utf8]{inputenc}
\usepackage[T1]{fontenc}
\usepackage{amsmath, amssymb, bm}
\usepackage{graphicx}
\usepackage{booktabs}
\usepackage{hyperref}
\usepackage{booktabs, tabularx, hyperref}
\usepackage{natbib}
\bibliographystyle{apalike}   
\hypersetup{colorlinks=true, urlcolor=blue,citecolor=blue!70!white, linkcolor=black}


% ===== TITLE =====
\title{Moved to Opportunity: The Long-Run Effects of Public
	Housing Demolition on Children (Chyn, 2018)}
\author{Magdalena Cortina \& Anel Rodríguez}
\institute{Texas A\&M University}
\date{\today}

\begin{document}
	
	% ---------- SLIDE 1 ----------
	\begin{frame}
		\titlepage
		
	\end{frame}
	

	
	% ---------- SLIDE 2----------
	\begin{frame}{Motivation}
		\textbf{Cities in the U.S have spent over \$6B demolishing public housing.} % and providing housing vouchers to displaced residents.	
% como poner un espacio aca?

		%\textbf{Public Housing Demolition in Chicago:}
		\begin{itemize}
			\item In the 1990s, Chicago had the third-largest public housing system, with 17 projects housing nearly 5\% of the population.
			\item Residents were very low-income, predominantly African American, and often single-parent, female-headed households.
			\item Demolition was driven by severe management failures and deteriorating infrastructure.
			\item Displaced families received vouchers or transfers within/across projects, with moving costs covered.
			\item Voucher subsidies covered the gap between rent (or Fair Market Rent) and 30\% of adjusted family income.
		\end{itemize}
	\end{frame}
	
		% ---------- SLIDE 3----------
	\begin{frame}{Motivation}

This paper estimates the long-run impacts of these demolitions on children in Chicago.

		\textbf{Main results:}
		\begin{itemize}
			\item Displaced households moved to areas with much lower poverty and violent crime.
			\item Displaced children show higher employment and earnings, fewer violent crime arrests, and lower dropout rates (for younger children).
			\item Both younger and older children benefit; estimated lifetime earnings rise by about \$45,000. % saber explicar bien este numero
			% I follow Chetty, Hendren, and Katz (2016) to predict the impact of this effect on lifetime earnings. Specifically, I use the following assumptions: (i) the 16 percent increase in annual income is constant over the life cycle; (ii) the profile of income for demolition participants follows the US population average; (iii) the real wage growth rate is 0.5 percent; and (iv) the discount rate is 3 percent. Based on these assumptions, relocating youth using vouchers would increase pretax lifetime income by about $45,000 (present value of about $12,000).
		\end{itemize}
	\end{frame}
	
	

	
	% ---------- SLIDE 5----------  tal vez esta se puede dividir en dos
	\begin{frame}{Data}
		Buildings (CHA):
		\begin{itemize}
			\item Demolitions during 1995-1998. 
			\item Buildings with +75 units.
			\item Projects where demolition may reflect unobserved tenant selection are excluded.
			\item Of these, 20 buildings were demolished (treated) and 33 remained open and serve as controls.
		\end{itemize}
		Households:
		\begin{itemize}
			%	\item The study links CHA building records with IDHS social assistance files (TANF/AFDC, Food Stamps, Medicaid, 1994–1997) to identify children living in public housing during demolition.
			\item Social assistance records with exact street addresss to identify households living in a buiding in the public housing project the year prior to demolition.
			\item 5,676 adults, covering at least 73\% of households in the demolition sample.
			\item \textbf{Focus:} 5,250 children from 2,767 households, ages 7-18 at demolition.
			\item This panel is merged with administrative data on labor market outcomes, assistance receipt, and criminal arrests.
		\end{itemize}
	\end{frame}
	
	
	
	
	% ---------- SLIDE 6----------
	\begin{frame}{Empirical Approach}
		\begin{itemize}
			\item Compare children in demolished buildings with those in non-demolished buildings within the same project.
			\item If assignment within projects is effectively random, differences reflect demolition effects.
			\item Model:
			\[
			Y_{it} = \alpha + \beta D_{b(i)} + \psi_{p(i)} + \varepsilon_{it}
			\]
			\item $D_{b(i)}$ indicates demolition; project fixed effects included; SEs clustered by building.
			\item Also estimate models with sex and age-group interactions (7–12 vs.\ 13–18).
			\item $\beta$ is causal if demolition was unrelated to resident traits.
		\end{itemize}
	\end{frame}
	
	
	
	
	
	% ---------- SLIDE 6.1----------
	\begin{frame}{Effect of Moving Out of Project-Based Public Housing}
		\begin{itemize}
			\item The study also uses building demolition as an instrument to estimate the “dose effect” of an additional year in high-rise public housing.
			
			\item 2SLS system:
			\[
			P_i = \gamma + \tau D_{b(i)} + \psi_{p(i)} + \eta_{it}
			\]
			\[
			Y_{it} = \pi + \theta P_i + \psi_{p(i)} + \varepsilon_{it}
			\]
			
			\item $P_i$ measures total years spent in project-based public housing (including pre-demolition years).
		\end{itemize}
	\end{frame}
	
	
	
	
	% ---------- SLIDE 6.2----------
%	\begin{frame}{Comparing Treated and Control Individuals Before Demolition}
	\begin{frame}{Identification Strategy}		
		\begin{itemize}
			\item The design requires that demolition was not correlated with children’s baseline characteristics.
			\item Treated and control children show no significant differences in prior crime, demographics, schooling, or by sex.
			\item Adults in demolition buildings also do not differ in prior criminal activity or labor market history.
		\end{itemize}
	\end{frame}
	
	
	
	
	
	% ---------- SLIDE 6.3----------
%	\begin{frame}{Testing for Attrition and Spatial Spillovers}
%		\begin{itemize}
%			\item Administrative data reduce attrition concerns, but bias could arise if displaced children leave the state and appear with zero earnings.
%			\item Using Grogger’s (2013) terminal-zero method, there is no evidence that displaced children attrit more than controls.
%			\item Tests for spatial spillovers, using indicators for living near demolition buildings, show no effects on labor market or welfare outcomes.
%		\end{itemize}
%	\end{frame}
	
	
	
	% ---------- SLIDE 7----------
	\begin{frame}{Main Results: Effects on Household Location}
		
		\centering
		\includegraphics[width=0.5\linewidth]{table4}
	%	\begin{itemize}
		%	\item Vouchers expand housing choice; baseline project areas had extremely high poverty (around 78\%).
		%	\item Address records allow tracking whether displaced households moved to lower-poverty neighborhoods.
		%	\item Displaced families did move to substantially better areas: three years after demolition they lived in tracts with about 21\% lower poverty and 42\% less violent crime.
		%	\item These neighborhood advantages decline over time, with much smaller differences by year eight.
	%	\end{itemize}
	\end{frame}
	
	
	
	% ---------- SLIDE 7.1----------
	\begin{frame}{Main Results: Effects on Labor Market Activity}
		
		\begin{itemize}
			\item Displaced children are more likely to be employed (about +4 p.p., 9\%) and earn more as adults (around +\$600, 16\%); they are also more likely to earn above \$14,000.
			\item Impacts are larger for girls, who show stronger gains in both employment and annual earnings; boys’ effects are positive but less precisely estimated.
			\item Effects vary with age at displacement: older children consistently benefit, while younger children see gains that grow as they age.
			\item By age 26, children displaced at young ages earn about \$3,000 more annually.
			\item Overall, benefits increase with longer exposure to improved neighborhood conditions.
		\end{itemize}
	\end{frame}
	
	
	
	% ---------- SLIDE 7.2----------
%	\begin{frame}{Main Results: Effects on Social Assistance and Crime}
%		\begin{itemize}
%			\item Demolition and relocation may influence welfare use and criminal behavior through improved neighborhood conditions.
%			\item Administrative assistance records allow tracking whether relocation changes participation in public programs.
%			\item Arrest data are used to assess impacts on criminal activity after moving to lower-crime areas.
%			\item Relocated youth have about 14\% fewer violent crime arrests, with larger reductions for males.
%		\end{itemize}
%	\end{frame}
	
	
	
	
	
	% ---------- SLIDE 7.3----------
	\begin{frame}{Main Results: Effects by Subgroup}
		\begin{itemize}
			\item Subgroup analysis shows consistently positive labor market impacts and reductions in violent arrests across baseline characteristics.
			\item Children from more disadvantaged backgrounds such as households with no working adults or projects with poverty above 70\%, experience especially large gains in employment and earnings.
			\item Overall, benefits appear broadly shared but strongest for the most disadvantaged children.
		\end{itemize}
	\end{frame}
	
	
	
	
	% ---------- SLIDE 7.4----------
	\begin{frame}{Main Results: Impact of Living in Public Housing on Labor Market Outcomes}
		\begin{itemize}
			\item Displaced children show clear improvements in employment and earnings as young adults.
			\item Demolition is used as an instrument to estimate the effect of additional years spent in public housing.
			\item Displaced children spent about 2.6 fewer years in public housing; each extra year of exposure reduces labor force participation by roughly 2 p.p. and annual earnings by about \$275.
		\end{itemize}
	\end{frame}
	
	
	
	% ---------- SLIDE 8----------
	\begin{frame}{Mediating Mechanisms}
		\begin{itemize}
			\item No evidence that demolition affected parents’ labor supply, ruling out a parental income channel.
			\item Moving to safer neighborhoods does not reduce teen arrests, so crime-related channels appear limited.
			\item Schooling matters: younger displaced children are less likely to drop out and show higher two-year college attendance, while older children show no schooling effects.
		\end{itemize}
	\end{frame}
	
	
	
	
	% ---------- SLIDE 9----------
	\begin{frame}{Cost-Benefit Analysis}
		\begin{itemize}
			\item Relocation raises adult earnings by about 16\%, implying roughly \$45,000 in lifetime gains (\$12,000 present value).
			\item Voucher programs cost far less than project-based housing; main expense is moving assistance (\$1,100 per family).
			\item For families with two children, benefits far exceed costs, yielding positive fiscal returns.
		\end{itemize}
	\end{frame}
	
	
	
	
	% ---------- SLIDE 10----------
	\begin{frame}{Conclusion}
		\begin{itemize}
			\item The study provides the first long-run causal evidence from Chicago’s 1990s demolitions.
			\item Displaced children show better labor market outcomes, especially those relocated at ages 7–12 (about +\$3,000 at age 26).
			\item They also have fewer violent crime arrests and lower high school dropout rates.
			\item Overall, relocation generates large benefits: lifetime earnings rise by roughly \$45,000 (\$12,000 present value), suggesting positive fiscal returns.
		\end{itemize}
	\end{frame}
	
	
=======
	
	% ---------- SLIDE 2----------
\begin{frame}{Introduction}
\begin{itemize}
    \item Cities have spent over \$6B demolishing public housing and providing vouchers.
    \item The paper estimates the long-run impacts of these demolitions on children in Chicago.
    \item Displaced households moved to areas with much lower poverty and violent crime.
    \item Displaced children show higher employment and earnings, fewer violent crime arrests, and lower dropout rates (for younger children).
    \item Both younger and older children benefit; estimated lifetime earnings rise by about \$45,000.
\end{itemize}
\end{frame}



    % ---------- SLIDE 3----------
\begin{frame}{Public Housing Demolition in Chicago}
\begin{itemize}
    \item In the 1990s, Chicago had the third-largest public housing system, with 17 projects housing nearly 5\% of the population.
    \item Residents were very low-income, predominantly African American, and often single-parent households.
    \item Demolition was driven by severe management failures and deteriorating infrastructure.
    \item Displaced families received Section 8 vouchers or transfers within/across projects, with moving costs covered.
    \item Voucher subsidies covered the gap between rent (or Fair Market Rent) and 30\% of adjusted family income.
\end{itemize}
\end{frame}




       % ---------- SLIDE 4----------
\begin{frame}{Expected Effects of Demolition on Children}
\begin{itemize}
    \item Displaced households may use vouchers to move to lower-poverty areas.
    \item Exposure to higher-income neighbors and peers may improve role models and information networks.
    \item Parents may access better job networks and invest more in children.
    \item Evidence on these mechanisms is mixed.
    \item Demolition may also affect children through school quality or less-dense private housing.
\end{itemize}
\end{frame}

>>>>>>> d953b3df2c8f51a1946a8721371c429e3f8e9d2a
	

    % ---------- SLIDE 5----------
\begin{frame}{Data Sources and Sample Construction}
\begin{itemize}
    \item The study links CHA building records with IDHS social assistance files (TANF/AFDC, Food Stamps, Medicaid, 1994–1997) to identify children living in public housing during demolition.
    \item Baseline characteristics and long-run outcomes are merged with IDES UI wage records (1995–2009), ISP arrest records (to 2009), and IDHS assistance files (1989–2009).
\end{itemize}
\end{frame}




     % ---------- SLIDE 5.1----------
\begin{frame}{Sample of Public Housing Buildings}
\begin{itemize}
    \item The analysis focuses on non-senior projects with demolitions during 1995–1998 (HOPE VI), limiting the sample to high-rise buildings (75+ units).
    \item Projects where demolition may reflect unobserved tenant selection are excluded.
    \item The final sample includes 53 high-rise buildings in 7 projects, with closure dates from Jacob (2004).
    \item Of these, 20 buildings were demolished (treated) and 33 remained open and serve as controls.
\end{itemize}
\end{frame}



  % ---------- SLIDE 5.2----------
\begin{frame}{Linking Households to the Public Housing System}
\begin{itemize}
    \item The study uses social assistance records with exact street addresses, identifying welfare recipients living in public housing in the year before closure. 
    \item These data include 5,676 adults, covering at least 73\% of households in the demolition sample.
    \item The focus is on children age 7–18 at demolition, with adult outcomes observed 3–14 years later. 
    \item The final sample includes 5,250 children from 2,767 households, forming a person-year panel through 2009.
    \item This panel is merged with administrative data on labor market outcomes, assistance receipt, and criminal arrests.
\end{itemize}
\end{frame}




     % ---------- SLIDE 6----------
\begin{frame}{Empirical Approach}
\begin{itemize}
    \item Compare children in demolished buildings with those in non-demolished buildings within the same project.
    \item If assignment within projects is effectively random, differences reflect demolition effects.
    \item Model:
    \[
        Y_{it} = \alpha + \beta D_{b(i)} + \psi_{p(i)} + \varepsilon_{it}
    \]
    \item $D_{b(i)}$ indicates demolition; project fixed effects included; SEs clustered by building.
    \item Also estimate models with sex and age-group interactions (7–12 vs.\ 13–18).
    \item $\beta$ is causal if demolition was unrelated to resident traits.
\end{itemize}
\end{frame}





        % ---------- SLIDE 6.1----------
\begin{frame}{Effect of Moving Out of Project-Based Public Housing}
\begin{itemize}
    \item The study also uses building demolition as an instrument to estimate the “dose effect” of an additional year in high-rise public housing.
    
    \item 2SLS system:
    \[
        P_i = \gamma + \tau D_{b(i)} + \psi_{p(i)} + \eta_{it}
    \]
    \[
        Y_{it} = \pi + \theta P_i + \psi_{p(i)} + \varepsilon_{it}
    \]

    \item $P_i$ measures total years spent in project-based public housing (including pre-demolition years).
\end{itemize}
\end{frame}




          % ---------- SLIDE 6.2----------
\begin{frame}{Comparing Treated and Control Individuals Before Demolition}
\begin{itemize}
    \item The design requires that demolition was not correlated with children’s baseline characteristics.
    \item Treated and control children show no significant differences in prior crime, demographics, schooling, or by sex.
    \item Adults in demolition buildings also do not differ in prior criminal activity or labor market history.
\end{itemize}
\end{frame}





           % ---------- SLIDE 6.3----------
\begin{frame}{Testing for Attrition and Spatial Spillovers}
\begin{itemize}
    \item Administrative data reduce attrition concerns, but bias could arise if displaced children leave the state and appear with zero earnings.
    \item Using Grogger’s (2013) terminal-zero method, there is no evidence that displaced children attrit more than controls.
    \item Tests for spatial spillovers, using indicators for living near demolition buildings, show no effects on labor market or welfare outcomes.
\end{itemize}
\end{frame}



     % ---------- SLIDE 7----------
\begin{frame}{Main Results: Effects on Household Location}
\begin{itemize}
    \item Vouchers expand housing choice; baseline project areas had extremely high poverty (around 78\%).
    \item Address records allow tracking whether displaced households moved to lower-poverty neighborhoods.
    \item Displaced families did move to substantially better areas: three years after demolition they lived in tracts with about 21\% lower poverty and 42\% less violent crime.
    \item These neighborhood advantages decline over time, with much smaller differences by year eight.
\end{itemize}
\end{frame}



     % ---------- SLIDE 7.1----------
\begin{frame}{Main Results: Effects on Labor Market Activity}
\begin{itemize}
    \item Displaced children are more likely to be employed (about +4 p.p., 9\%) and earn more as adults (around +\$600, 16\%); they are also more likely to earn above \$14,000.
    \item Impacts are larger for girls, who show stronger gains in both employment and annual earnings; boys’ effects are positive but less precisely estimated.
    \item Effects vary with age at displacement: older children consistently benefit, while younger children see gains that grow as they age.
    \item By age 26, children displaced at young ages earn about \$3,000 more annually.
    \item Overall, benefits increase with longer exposure to improved neighborhood conditions.
\end{itemize}
\end{frame}



 % ---------- SLIDE 7.2----------
\begin{frame}{Main Results: Effects on Social Assistance and Crime}
\begin{itemize}
    \item Demolition and relocation may influence welfare use and criminal behavior through improved neighborhood conditions.
    \item Administrative assistance records allow tracking whether relocation changes participation in public programs.
    \item Arrest data are used to assess impacts on criminal activity after moving to lower-crime areas.
    \item Relocated youth have about 14\% fewer violent crime arrests, with larger reductions for males.
\end{itemize}
\end{frame}



    

     % ---------- SLIDE 7.3----------
\begin{frame}{Main Results: Effects by Subgroup}
\begin{itemize}
    \item Subgroup analysis shows consistently positive labor market impacts and reductions in violent arrests across baseline characteristics.
    \item Children from more disadvantaged backgrounds such as households with no working adults or projects with poverty above 70\%, experience especially large gains in employment and earnings.
    \item Overall, benefits appear broadly shared but strongest for the most disadvantaged children.
\end{itemize}
\end{frame}


    

         % ---------- SLIDE 7.4----------
\begin{frame}{Main Results: Impact of Living in Public Housing on Labor Market Outcomes}
\begin{itemize}
    \item Displaced children show clear improvements in employment and earnings as young adults.
    \item Demolition is used as an instrument to estimate the effect of additional years spent in public housing.
    \item Displaced children spent about 2.6 fewer years in public housing; each extra year of exposure reduces labor force participation by roughly 2 p.p. and annual earnings by about \$275.
\end{itemize}
\end{frame}



         % ---------- SLIDE 8----------
\begin{frame}{Mediating Mechanisms}
\begin{itemize}
    \item No evidence that demolition affected parents’ labor supply, ruling out a parental income channel.
    \item Moving to safer neighborhoods does not reduce teen arrests, so crime-related channels appear limited.
    \item Schooling matters: younger displaced children are less likely to drop out and show higher two-year college attendance, while older children show no schooling effects.
\end{itemize}
\end{frame}




             % ---------- SLIDE 9----------
\begin{frame}{Cost-Benefit Analysis}
\begin{itemize}
    \item Relocation raises adult earnings by about 16\%, implying roughly \$45,000 in lifetime gains (\$12,000 present value).
    \item Voucher programs cost far less than project-based housing; main expense is moving assistance (\$1,100 per family).
    \item For families with two children, benefits far exceed costs, yielding positive fiscal returns.
\end{itemize}
\end{frame}




         % ---------- SLIDE 10----------
\begin{frame}{Conclusion}
\begin{itemize}
    \item The study provides the first long-run causal evidence from Chicago’s 1990s demolitions.
    \item Displaced children show better labor market outcomes, especially those relocated at ages 7–12 (about +\$3,000 at age 26).
    \item They also have fewer violent crime arrests and lower high school dropout rates.
    \item Overall, relocation generates large benefits: lifetime earnings rise by roughly \$45,000 (\$12,000 present value), suggesting positive fiscal returns.
\end{itemize}
\end{frame}



	
	\begin{frame}{}
		
		\centering
	\large
		Thank you!
		
	\end{frame}
	
	
	
	
	
\end{document}
