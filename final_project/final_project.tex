% ===== Document class & geometry =====
\documentclass{article}
\usepackage[top=1in, bottom=1in, left=1in, right=1in]{geometry}

% ===== Encoding & language =====
\usepackage[utf8]{inputenc} % <-- swap utf8x -> utf8
\usepackage[english]{babel}

% ===== Math =====
\usepackage{amsmath, amsfonts, amssymb, bm}
\usepackage{amsthm}

% ===== Tables & floats =====
\usepackage[table]{xcolor}
\usepackage{array, tabularx, booktabs, longtable, makecell, multirow}
% \usepackage{tabu} % <-- REMOVE: abandoned & buggy
\usepackage{caption}
\usepackage{subcaption} % keep this; DO NOT also load floatrow
% \usepackage[capposition=top]{floatrow} % <-- REMOVE: conflicts with subcaption

% ===== Graphics =====
\usepackage{graphicx}
\usepackage{pdflscape}
\usepackage{pdfpages}
\usepackage{rotating}
\usepackage{epstopdf}

% ===== Text, layout, headers/footers =====
\usepackage{setspace}
\usepackage{ragged2e}
\usepackage{fancyhdr}
\usepackage[normalem]{ulem}
\usepackage{enumerate}
\usepackage{framed}
\usepackage{comment}
\usepackage{titlesec}
\usepackage{titletoc} % load once
\usepackage{accents}
\usepackage{stackengine}
\usepackage{footmisc} % choose ONE footnote package
% \usepackage{footnote} % <-- REMOVE to avoid conflicts

% ===== Links (load near the end) =====
\usepackage[colorlinks=true,linkcolor=darkgray,citecolor=darkgray,urlcolor=darkgray,anchorcolor=darkgray]{hyperref}

% ===== Bibliography (natbib + BibTeX) =====
\usepackage[sort&compress]{natbib}
% \usepackage{bibunits} % <-- Usually unnecessary; remove unless you truly need per-section bibs

% ===== Colors & watermark (optional) =====
\definecolor{lightgray}{RGB}{220,220,220}
\definecolor{dimgray}{RGB}{105,105,105}
% \usepackage[printwatermark]{xwatermark} % optional; safe to keep disabled

% ===== Section formats =====
\titlespacing{\section}{.2pt}{1ex}{1ex}
\titleformat{\section}{\centering \normalsize \bfseries}{\thesection.}{0em}{}
\renewcommand{\thesection}{\arabic{section}}
\titleformat{\subsection}{\flushleft \normalsize \bfseries}{\thesubsection}{0em}{}
\renewcommand{\thesubsection}{\arabic{section}.\arabic{subsection}}

% ===== Theorems =====
\newtheoremstyle{mytheoremstyle}
{\topsep}{\topsep}{\color{black}}{}{\itshape\color{dimgray}}{.}{.5em}{}
\theoremstyle{mytheoremstyle}
\newtheorem{assumption}{Assumption}
\renewcommand\theassumption{\arabic{assumption}}
\newtheorem{assumptiona}{Assumption}
\renewcommand\theassumptiona{\arabic{assumptiona}a}
\newtheorem{assumptionb}{Assumption}
\renewcommand\theassumptionb{\arabic{assumptionb}b}
\newtheorem{assumptionc}{Assumption}
\renewcommand\theassumptionc{\arabic{assumptionc}c}
\newtheorem{lemma}{Lemma}
\newtheorem{proposition}{Proposition}
\newtheorem{corollary}{Corollary}

% ===== Math operators & handy commands =====
\DeclareMathOperator{\cov}{Cov}
\DeclareMathOperator{\sign}{sgn}
\DeclareMathOperator{\var}{Var}
\DeclareMathOperator{\plim}{plim}
\DeclareMathOperator*{\argmin}{arg\,min}
\DeclareMathOperator*{\argmax}{arg\,max}
\newcommand\independent{\protect\mathpalette{\protect\independenT}{\perp}}
\def\independenT#1#2{\mathrel{\rlap{$#1#2$}\mkern2mu{#1#2}}}
\newcommand{\overbar}[1]{\mkern 1.5mu\overline{\mkern-1.5mu#1\mkern-1.5mu}\mkern 1.5mu}
\newcommand{\equald}{\ensuremath{\overset{d}{=}}}
\newcommand\barbelow[1]{\stackunder[1.2pt]{$#1$}{\rule{1ex}{.085ex}}}

% ===== Captions =====
\captionsetup[table]{skip=10pt}
\captionsetup[figure]{labelfont={bf},name={Figure},labelsep=period}
\renewcommand{\thefigure}{\arabic{figure}}
\captionsetup[table]{labelfont={bf},name={Table},labelsep=period}
\renewcommand{\thetable}{\arabic{table}}

% ===== Paragraphing, spacing, column types =====
\setlength{\parindent}{24pt} % (you had both 22pt and 24pt—choose one)
\setlength{\parskip}{5pt}
\newcolumntype{L}[1]{>{\raggedright\let\newline\\\arraybackslash\hspace{0pt}}m{#1}}
\newcolumntype{C}[1]{>{\centering\let\newline\\\arraybackslash\hspace{0pt}}m{#1}}
\newcolumntype{R}[1]{>{\raggedleft\let\newline\\\arraybackslash\hspace{0pt}}m{#1}}

% ===== REMOVE the risky patch (likely cause of \endgroup) =====
% \makeatletter
% \pretocmd\start@align{ \let\everycr\CT@everycr \CT@start }{}{}
% \apptocmd{\endalign}{\CT@end}{}{}
% \makeatother


\begin{document}


\title{\Large \textbf{ECON 603 - Research Proposal}}

\author{Magdalena Cortina\thanks{Email: mcortinat@tamu.edu.}} 
\date{\today}

\maketitle
\thispagestyle{empty} 
\doublespacing
\thispagestyle{empty} 

\vspace{-10mm}
\pagenumbering{arabic}

\doublespacing

\section{ Literature Review}

Role models, as delineated in the literature, are defined along a spectrum of characteristics predominantly grounded in social psychology. These individuals influence the behavior of others merely through exposure \citep{merton_social_1968}. Observing the success of a model’s behavior substantially enhances motivation, emerging as a pivotal factor that determines the effectiveness of behavior modeling \citep{bandura_social_1977}. Moreover, a role model typically embodies the experience and expertise that an observer lacks—or perceives themselves to lack—thereby providing a pathway for learning through observation and comparison \citep{kemper_reference_1968}. The observer attentively studies the model’s behavior within its original context and subsequently applies the acquired insights across broader situations \citep{brophy_child_1977}.

Numerous studies have extensively explored the impact of role models across diverse contexts, revealing insightful observations. \citet{nguyen_information_nodate} presents findings from Madagascar, indicating that the exposure of fourth-grade students and their parents to role models from comparable socioeconomic backgrounds amplifies perceived returns to education. This exposure intensifies incentives for education among families who underestimate the actual benefits. Furthermore, evidence from Brazil underscores the influence of television in shaping perceptions. Exposure to soap operas featuring main characters with either one child or none significantly diminishes women’s fertility rates \citep{la_ferrara_soap_2012}.

In rural Ethiopia, \citet{bernard_future_nodate} conducted an experimental study demonstrating substantial impacts. Individuals randomly exposed to documentaries depicting successful agricultural or business endeavors within communities similar to their own exhibited heightened aspirations compared to the control group. This surge in aspirations translated into increased savings, greater credit utilization, enhanced educational pursuits for children, and higher expenditure on education.
Additionally, \citet{riley_role_2024} provides evidence from Uganda showcasing the remarkable influence of female role models on academic performance, particularly among women. Exposure to a film featuring a female role model substantially elevated academic achievement, underscoring the pronounced effect it had, especially for female students.

Numerous studies have explored how female role models influence female students’ inclination toward STEM careers. In South Korea, \citet{lim_persistent_2020} discovered that seventh-grade female students taught by female teachers outperformed their counterparts taught by male teachers on standardized tests. This advantage persisted for up to five years, correlating with an increased likelihood of pursuing STEM careers. Similarly, in France, \citet{breda_female_2020} found that exposure to external female role models in 12th grade elevated the probability of female students entering male-dominated STEM fields by 30\% within a year. In Chile, \citet{paredes_teacher_2014} identified a significant impact of having a female mathematics teacher in eighth grade, resulting in improved mathematics scores on the SIMCE exam—equivalent to almost a quarter of the gender gap in math performance—indicating a role-model effect.

At the undergraduate level, \citet{carrell_sex_2010} find that having a female teacher in introductory math and science courses at the U.S. Air Force Academy increases the probability that top female students pursue STEM majors by 26 percentage points compared with having exclusively male teachers. All of this literature converges on an important point: the effect of having a same-sex mentor is especially strong for women, while it remains limited for men \citep{carrell_sex_2010,paredes_teacher_2014}.

\citet{porter_gender_2020} studied the impact of exposure to female role models in introductory economics courses at an American university. Female alumni of the economics major delivered neutral, gender-oriented speeches explaining how their choice of an economics degree influenced their professional trajectories and contributed to their success. Employing a difference-in-differences strategy, the study revealed a notable and statistically significant eight-percentage-point increase from a 9\% baseline in female students’ inclination to major in economics following these interventions. The research indicates that the mechanism driving this shift is primarily inspirational rather than informational. The study underscores the efficacy of a straightforward and cost-effective intervention that can encourage women to pursue fields traditionally dominated by men, which are often associated with higher earning potential.

 \citet{griffith_role_2021} examine whether female graduate teaching assistants (TAs) can serve as effective role models for female undergraduates in engineering. Using data from first-year engineering students who were randomly assigned to TAs, they find limited evidence that female TAs improve women’s course grades or persistence in engineering. However, they do find that female students assigned to female TAs are significantly more likely to choose high-earning engineering fields, especially when the classroom environment includes a female instructor and a higher share of female peers. These findings suggest that gender-matching effects in the classroom may influence female students’ confidence and field choices rather than immediate academic performance, highlighting the potential of near-peer female mentors to encourage women’s entry into competitive and lucrative disciplines.
 
<<<<<<< HEAD
 \section{ The Chilean Context}
 
 \subsection{ Chilean University Admission System}
The Chilean university admission system is administered by the \textit{Department of Educational Evaluation, Measurement, and Registration} (DEMRE, for its initials in Spanish), an entity of the University of Chile. Thirty-six universities affiliated with the system—both public and private—participate in this program, selecting their students each year through a standardized test. The admission test\footnote{Although the test format has changed over the years, the core structure of the admission process has remained the same.}, together with high school grades, determines a weighted score with which students submit a ranked list of ten program choices in order of preference. Although students are uncertain whether they will be admitted to their preferred program, they often rely on the previous year's cutoff score as a reference.

Given this, students must choose their major before entering college, once they receive their test scores. Unlike the American system, there is no period during the first years of college when they can decide which major to pursue. University programs in Chile typically last about five years: students obtain a bachelor’s degree in the fourth year and a professional title upon completion.

\subsection{ The Commercial Engineering Program}
In Chile, economics coexists with other disciplines—particularly business administration—within a single professional program known as Commercial Engineering. Therefore, economics itself is not a standalone professional degree but rather a specialization within this program, corresponding to a bachelor's degree. This distinction is important because in Chile professional titles tend to carry more weight than bachelor’s degrees, and there is no professional title of Economist.

Although business administration and economics are quite different fields, they share a common academic curriculum during the first two years of the program, after which students begin to specialize by choosing one of the two majors. Before making this choice, the common curriculum at the institution from which the data used in this study are drawn, includes the courses listed in Table 1.

At the end of the fourth semester, once these courses have ideally been completed, students select their major. Those who choose Economics enroll in Mathematical Economics during the fifth semester, which is the first course specific to the major. Only the courses shown in bold in Table 1 serve as prerequisites for this subject: Introduction to Economics, Microeconomics I, Algebra I, Calculus I, and Calculus II. It is worth noting that students are not asked at the beginning of the program which major they intend to pursue.

	
	\begin{table}[ht]
		\centering
		\caption{Commercial Engineering Academic Program, by semester}
		\label{tab:program}
		% Define a ragged-right X column with small padding
		\newcolumntype{Y}{>{\raggedright\arraybackslash}X}
		% One number column + 7 course columns (adjust the 7 to your needs)
		\tiny
		\setlength{\extrarowheight}{4pt}
		\begin{tabularx}{\textwidth}{>{\centering\arraybackslash}p{0.9em} | Y |Y| Y |Y |Y |Y| Y|}
			\toprule
			\multicolumn{1}{c}{Semester} & \multicolumn{7}{c}{Courses} \\
			\midrule
			1 &
			Writing &
			Introduction to Business &
			\textbf{Introduction to Economics} &
			\textbf{Calculus I} &
			\textbf{Algebra I} &
			& \\ \hline
			
			2 &
			Verbal Reasoning &
			Entrepreneurship Workshop &
			Accounting I &
			\textbf{Calculus II} &
			Macroeconomics I &
			& \\ \hline
			
			3 &
			Political Analysis &
			Science I &
			Accounting II &
			Calculus III &
			\textbf{Microeconomics I} &
			People and Organizations &
			Leadership I \\ \hline
			
			4 &
			Logic and Argument Theory &
			Science II &
			Microeconomics II &
			Algebra II &
			Statistics I &
			Firm Environment &
			Oral Expression \\ 
			\bottomrule
		\end{tabularx}
		
		\vspace{3pt}
		\footnotesize \emph{Notes:} Bold courses indicate prerequisites for \emph{Mathematical Economics} (the first Economics-major course).
	\end{table}
	

\subsection{ Teaching Assistants}
Teaching assistants are typically students who demonstrate exceptional mastery of the subject they support. While not considered experts, they possess in-depth knowledge of the course content and excel at explaining exercises and concepts clearly and systematically during complementary sessions under the supervision of course instructors. Their responsibilities include leading additional sessions alongside regular classes and being available to answer student questions throughout the semester. The selection process for teaching assistants varies; professors often choose them either through an online application system or by direct appointment.

There are two main types of teaching assistants: those who regularly lead classes and those responsible for grading assessments such as exams, quizzes, and assignments. Typically, each course has two assistants who share duties related to teaching and grading, and they often help supervise students during examinations. Most teaching assistants from the sample are students from the same university. This shared educational background, combined with the relative closeness in age between assistants and students, fosters a more personal connection and increases the likelihood of establishing a stronger rapport than that usually formed with traditional instructors.

\section{ Data}

For this study, I had access to administrative data from a private university in Santiago, Chile. The dataset includes course enrollment records linked to their respective class numbers. The decision variable for choosing the Economics major is enrollment in the course Mathematical Economics (ECO301), which is the first course of the major and, based on academic progression, corresponds to the fifth semester. Accordingly, the sample was restricted to students who had passed all prerequisite courses required to enroll in Mathematical Economics—that is, those eligible to choose this first course of the Bachelor’s in Economics. 

Based on these criteria, the final sample consists of 2,776 students from cohorts 2013-2019\footnote{I'm on the process of updating the data to add more cohorts. The study was already submitted to Huron and I'm waiting on that to be able to receive the rest of the data.}. From these sample, 40\% are women and 10\% choose the economics major\footnote{This is something that will change with the updated data. The econ major has been growing exponentially.}. 88\% of the students went to a private school. [still working on a table on descriptive statistics]

A list of all the variables that I have in the data is the following:
\begin{itemize}
	\item Student ID
	\item Student birth year
	\item Student gender
	\item Highschool municipality
	\item Type of highschool
	\item College program
	\item Cohort
	\item Highschool grades average
	\item PSU scores (standarized test to apply to universities)
	\item Weighted PSU score
	\item Ranking PSU score
	\item Average PSU score
	\item Fifth year College program (master's)
	\item College application type
	\item College status
	\item Reason for inactivity (if inactive)
	\item TA id
	\item TA type (instructor / grader)
	\item TA gender
	\item TA is in the Econ major
	\item TA evaluation of the course
	\item Professor ID
	\item Section id
	\item Course initials
	\item Course grade
	\item Campus
	\item Semester
	\item Professor gender
	\item Professor type (hour / plant)
	\item Professor evaluation of the course
\end{itemize}

=======
 \noindent \textbf{[JLG: Approved. The Madagascar reference is broken, but that is a minor issue.]}
>>>>>>> 2504b019ca22b7918d9d5007882449b360783d62

\bibliographystyle{chicago}
\bibliography{women_in_econ}

\end{document}
