% ===== Document class & geometry =====
\documentclass{article}
\usepackage[top=1in, bottom=1in, left=1in, right=1in]{geometry}

% ===== Encoding & language =====
\usepackage[utf8]{inputenc} % <-- swap utf8x -> utf8
\usepackage[english]{babel}

% ===== Math =====
\usepackage{amsmath, amsfonts, amssymb, bm}
\usepackage{amsthm}

% ===== Tables & floats =====
\usepackage[table]{xcolor}
\usepackage{array, tabularx, booktabs, longtable, makecell, multirow}
% \usepackage{tabu} % <-- REMOVE: abandoned & buggy
\usepackage{caption}
\usepackage{subcaption} % keep this; DO NOT also load floatrow
% \usepackage[capposition=top]{floatrow} % <-- REMOVE: conflicts with subcaption

% ===== Graphics =====
\usepackage{graphicx}
\usepackage{pdflscape}
\usepackage{pdfpages}
\usepackage{rotating}
\usepackage{epstopdf}

% ===== Text, layout, headers/footers =====
\usepackage{setspace}
\usepackage{ragged2e}
\usepackage{fancyhdr}
\usepackage[normalem]{ulem}
\usepackage{enumerate}
\usepackage{framed}
\usepackage{comment}
\usepackage{titlesec}
\usepackage{titletoc} % load once
\usepackage{accents}
\usepackage{stackengine}
\usepackage{footmisc} % choose ONE footnote package
% \usepackage{footnote} % <-- REMOVE to avoid conflicts

% ===== Links (load near the end) =====
\usepackage[colorlinks=true,linkcolor=darkgray,citecolor=darkgray,urlcolor=darkgray,anchorcolor=darkgray]{hyperref}

% ===== Bibliography (natbib + BibTeX) =====
\usepackage[sort&compress]{natbib}
% \usepackage{bibunits} % <-- Usually unnecessary; remove unless you truly need per-section bibs

% ===== Colors & watermark (optional) =====
\definecolor{lightgray}{RGB}{220,220,220}
\definecolor{dimgray}{RGB}{105,105,105}
% \usepackage[printwatermark]{xwatermark} % optional; safe to keep disabled

% ===== Section formats =====
\titlespacing{\section}{.2pt}{1ex}{1ex}
\titleformat{\section}{\centering \normalsize \bfseries}{\thesection.}{0em}{}
\renewcommand{\thesection}{\arabic{section}}
\titleformat{\subsection}{\flushleft \normalsize \bfseries}{\thesubsection}{0em}{}
\renewcommand{\thesubsection}{\arabic{section}.\arabic{subsection}}

% ===== Theorems =====
\newtheoremstyle{mytheoremstyle}
{\topsep}{\topsep}{\color{black}}{}{\itshape\color{dimgray}}{.}{.5em}{}
\theoremstyle{mytheoremstyle}
\newtheorem{assumption}{Assumption}
\renewcommand\theassumption{\arabic{assumption}}
\newtheorem{assumptiona}{Assumption}
\renewcommand\theassumptiona{\arabic{assumptiona}a}
\newtheorem{assumptionb}{Assumption}
\renewcommand\theassumptionb{\arabic{assumptionb}b}
\newtheorem{assumptionc}{Assumption}
\renewcommand\theassumptionc{\arabic{assumptionc}c}
\newtheorem{lemma}{Lemma}
\newtheorem{proposition}{Proposition}
\newtheorem{corollary}{Corollary}

% ===== Math operators & handy commands =====
\DeclareMathOperator{\cov}{Cov}
\DeclareMathOperator{\sign}{sgn}
\DeclareMathOperator{\var}{Var}
\DeclareMathOperator{\plim}{plim}
\DeclareMathOperator*{\argmin}{arg\,min}
\DeclareMathOperator*{\argmax}{arg\,max}
\newcommand\independent{\protect\mathpalette{\protect\independenT}{\perp}}
\def\independenT#1#2{\mathrel{\rlap{$#1#2$}\mkern2mu{#1#2}}}
\newcommand{\overbar}[1]{\mkern 1.5mu\overline{\mkern-1.5mu#1\mkern-1.5mu}\mkern 1.5mu}
\newcommand{\equald}{\ensuremath{\overset{d}{=}}}
\newcommand\barbelow[1]{\stackunder[1.2pt]{$#1$}{\rule{1ex}{.085ex}}}

% ===== Captions =====
\captionsetup[table]{skip=10pt}
\captionsetup[figure]{labelfont={bf},name={Figure},labelsep=period}
\renewcommand{\thefigure}{\arabic{figure}}
\captionsetup[table]{labelfont={bf},name={Table},labelsep=period}
\renewcommand{\thetable}{\arabic{table}}

% ===== Paragraphing, spacing, column types =====
\setlength{\parindent}{24pt} % (you had both 22pt and 24pt—choose one)
\setlength{\parskip}{5pt}
\newcolumntype{L}[1]{>{\raggedright\let\newline\\\arraybackslash\hspace{0pt}}m{#1}}
\newcolumntype{C}[1]{>{\centering\let\newline\\\arraybackslash\hspace{0pt}}m{#1}}
\newcolumntype{R}[1]{>{\raggedleft\let\newline\\\arraybackslash\hspace{0pt}}m{#1}}

% ===== REMOVE the risky patch (likely cause of \endgroup) =====
% \makeatletter
% \pretocmd\start@align{ \let\everycr\CT@everycr \CT@start }{}{}
% \apptocmd{\endalign}{\CT@end}{}{}
% \makeatother


\begin{document}


\title{\Large \textbf{ECON 603 - Research Proposal}}

\author{Magdalena Cortina\thanks{Email: mcortinat@tamu.edu.}} 
\date{\today}

\maketitle
\thispagestyle{empty} 
\doublespacing
\thispagestyle{empty} 

\vspace{-10mm}
\pagenumbering{arabic}

\doublespacing

\section*{Literature Review}

Role models, as delineated in the literature, are defined along a spectrum of characteristics predominantly grounded in social psychology. These individuals influence the behavior of others merely through exposure \citep{merton_social_1968}. Observing the success of a model’s behavior substantially enhances motivation, emerging as a pivotal factor that determines the effectiveness of behavior modeling \citep{bandura_social_1977}. Moreover, a role model typically embodies the experience and expertise that an observer lacks—or perceives themselves to lack—thereby providing a pathway for learning through observation and comparison \citep{kemper_reference_1968}. The observer attentively studies the model’s behavior within its original context and subsequently applies the acquired insights across broader situations \citep{brophy_child_1977}.

Numerous studies have extensively explored the impact of role models across diverse contexts, revealing insightful observations. \citet{nguyen_information_nodate} presents findings from Madagascar, indicating that the exposure of fourth-grade students and their parents to role models from comparable socioeconomic backgrounds amplifies perceived returns to education. This exposure intensifies incentives for education among families who underestimate the actual benefits. Furthermore, evidence from Brazil underscores the influence of television in shaping perceptions. Exposure to soap operas featuring main characters with either one child or none significantly diminishes women’s fertility rates \citep{la_ferrara_soap_2012}.

In rural Ethiopia, \citet{bernard_future_nodate} conducted an experimental study demonstrating substantial impacts. Individuals randomly exposed to documentaries depicting successful agricultural or business endeavors within communities similar to their own exhibited heightened aspirations compared to the control group. This surge in aspirations translated into increased savings, greater credit utilization, enhanced educational pursuits for children, and higher expenditure on education.
Additionally, \citet{riley_role_2024} provides evidence from Uganda showcasing the remarkable influence of female role models on academic performance, particularly among women. Exposure to a film featuring a female role model substantially elevated academic achievement, underscoring the pronounced effect it had, especially for female students.

Numerous studies have explored how female role models influence female students’ inclination toward STEM careers. In South Korea, \citet{lim_persistent_2020} discovered that seventh-grade female students taught by female teachers outperformed their counterparts taught by male teachers on standardized tests. This advantage persisted for up to five years, correlating with an increased likelihood of pursuing STEM careers. Similarly, in France, \citet{breda_female_2020} found that exposure to external female role models in 12th grade elevated the probability of female students entering male-dominated STEM fields by 30\% within a year. In Chile, \citet{paredes_teacher_2014} identified a significant impact of having a female mathematics teacher in eighth grade, resulting in improved mathematics scores on the SIMCE exam—equivalent to almost a quarter of the gender gap in math performance—indicating a role-model effect.
At the undergraduate level, \citet{carrell_sex_2010} find that having a female teacher in introductory math and science courses at the U.S. Air Force Academy increases the probability that top female students pursue STEM majors by 26 percentage points compared with having exclusively male teachers. All of this literature converges on an important point: the effect of having a same-sex mentor is especially strong for women, while it remains limited for men \citep{carrell_sex_2010,paredes_teacher_2014}.
\citet{porter_gender_2020} studied the impact of exposure to female role models in introductory economics courses at an American university. Female alumni of the economics major delivered neutral, gender-oriented speeches explaining how their choice of an economics degree influenced their professional trajectories and contributed to their success. Employing a difference-in-differences strategy, the study revealed a notable and statistically significant eight-percentage-point increase from a ()9\% baseline) in female students’ inclination to major in economics following these interventions. The research indicates that the mechanism driving this shift is primarily inspirational rather than informational. The study underscores the efficacy of a straightforward and cost-effective intervention that can encourage women to pursue fields traditionally dominated by men, which are often associated with higher earning potential.

 \citet{griffith_role_2021} examine whether female graduate teaching assistants (TAs) can serve as effective role models for female undergraduates in engineering. Using data from first-year engineering students who were randomly assigned to TAs, they find limited evidence that female TAs improve women’s course grades or persistence in engineering. However, they do find that female students assigned to female TAs are significantly more likely to choose high-earning engineering fields, especially when the classroom environment includes a female instructor and a higher share of female peers. These findings suggest that gender-matching effects in the classroom may influence female students’ confidence and field choices rather than immediate academic performance, highlighting the potential of near-peer female mentors to encourage women’s entry into competitive and lucrative disciplines.

\bibliographystyle{chicago}
\bibliography{women_in_econ}

\end{document}
